
\chapter{Change Log}

\subsection*{Version 2.1.0}
In addition to the usual internal improvements, this minor version updates brings a couple of new features and a change of one header file location:
\begin{itemize}
 \item scale(): Moved from file \lstinline|viennagrid/algorithm/scale.hpp| to \lstinline|viennagrid/algorithm/geometric_transform.hpp|.
 \item Refinement: Added support for specifying the location of the new vertex for each edge to be refined.
 \item Refinement: Added optional vertex copy map, which returns the new vertex in the refined mesh given a vertex from the original mesh.
 \item Refinement: Added refinement towards a hyperplane, so that no element of the refined mesh is intersected by the hyperplane.
 \item Added thin mesh configurations, where only the cells and the vertices are stored. This leads to a much lower memory footprint provided that edges, facets, etc. are not needed.
 \item Added \lstinline|segmented_mesh| class, which encapsulates one mesh and an associated segmentation.
 \item Added support for named segments. Rather than only providing a numeric ID, segments can be identified by a string.
 \item Removed \lstinline|plc_2d_*| typedefs since they might cause confusion. Introduced boundary representation typedefs \lstinline|brep_*d_*| instead.
 \item IO: Added reader for .bnd files, which are created by Synopsys tools.
 \item IO: Added reader for .tess files (produced by Neper polycrystal generation and meshing library).
 \item IO: Added writer for .mphtxt files (consumed by COMSOL).
 \item IO: Added preliminary writer for .vmesh legacy files.
 \item IO: VTK writer now writes only one .vtu file if the segmentation is empty or only one segment is present.
 \item Added convenience routine for copying elements based on an element iterator range or element handle iterator range.
 \item Algorithms: Added boundary/hull extraction.
 \item Algorithms: Generalized geometric transformations (scale, affine transform).
 \item Algorithms: Added bounding box computation, normal vector computation, and determinant of points/vectors.
 \item Algorithms: Added inclusion tests for points inside triangles or tetrahedra.
 \item Algorithms: Added distance between segment boundary and line.
 \item Bugfix: Fixed logic error in computation of Voronoi quantities.
 \item Bugfix: Compilation error for \lstinline|closest_point()| for boundary distance.
 \item Bugfix: Element handles were not resolved correctly when elements were deleted from a mesh.
\end{itemize}

\subsection*{Version 2.0.0}
The ViennaGrid internals have been completely redesigned for higher flexibility.
Some rather significant adjustments to the user API were necessary.
\begin{itemize}
 \item Renamed the old \lstinline|domain_t| to \lstinline|mesh| in order to avoid ambiguities with the mathematical \emph{problem domain}.
 \item Replaced \lstinline|viennagrid::ncells<>()| with \lstinline|viennagrid::elements<>()| to obtain range objects.
 \item As a consequence of moving away from \lstinline|ncells<>|, now element tags are used instead of the topologic dimension to select elements.
 \item Added support for two dynamic element types: polygon and PLCs.
 \item Added support for neighbor iteration. This way one no longer needs to code the boundary/coboundary iterations by hand.
 \item Added support for multiple segmentations. This is a generalization of the old segment concept, where elements could be part of at most one segment.
 \item New algorithms: angles, intersection, scaling, seed point segmenting.
 \item Accessors are now consistently used for accessing quantities rather than ViennaData. This makes the implementation more generic and provides better support for user storage.
 \item New storage layer, re-wrote most of the internals.
\end{itemize}

\subsection*{Version 1.0.1}
This is a maintenance release, mostly fixing minor compilation problems on some compilers and operating systems. Other notable changes:
\begin{itemize}
  \item Added \lstinline|distance()| function for computing the distance between points, cell, etc.
  \item Voronoi quantities can now also be accessed in a more fine-grained manner: Volume and contributions for each cell attached to a vertex or edge.
  \item Added quantity transfer: Interpolates quantities on a $m$-cells can be transferred to $n$-cells. Both $m<n$ and $n>m$ are supported.
\end{itemize}

\subsection*{Version 1.0.0}
First release

\chapter{IO} \label{chap:io}
Due to the high number of vertices and cells in typical meshes,
a manual setup of a domain in code is soon inefficient. Therefore,
the geometric locations as well as topological connections are typically stored in mesh files.

Unfortunately, the number of different mesh formats seems to grow in a rather uncontrolled manner.
Thus, as soon as a reader for a particular mesh file format is implemented, two new formats have emerged.
In order not to give birth to another mesh file format, {\ViennaGrid} does not bring its own file format. 
Instead, the library mainly relies on the XML-based VTK file format \cite{VTK,VTKfileformat}.

\TIP{Let us know about your favorite file format(s)!
 Send an email to our mailinglist: \texttt{viennagrid-support$@$lists.sourceforge.net}.
 It increases the chances of having a reader and/or writer included in the next {\ViennaGrid} release.}




\section{Readers}

\TIP{File reader contributions are always welcome!}

 \subsection{Netgen}

 \subsection{VTK}



\section{Writers}

\TIP{File writer contributions are always welcome!}

 \subsection{OpenDX}

 \subsection{VTK}

